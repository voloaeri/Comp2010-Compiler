\section{Dynamic folding example}
\label{sec:example}

The following example illustrates how well our optimization performs. It implies dynamic and constant folding.\\

Given the following Java code:
 
\begin{lstlisting}[language=Java]
int a = 42;
int b = (a + 764) * 3;
a = 22;
return b + 1234 - a; 
\end{lstlisting}

Our optimization algorithm is able to perform all calculations at compile time and replaces almost all of the instructions of the unoptimized code with a single push instruction:

\newsavebox{\unoptimized}
\begin{lrbox}{\unoptimized}
\begin{lstlisting}
0: bipush        42
2: istore_1
3: iload_1
4: sipush        764
7: iadd
8: iconst_3
9: imul
10: istore_2
11: bipush        22
13: istore_1
14: iload_2
15: sipush        1234
18: iadd
19: iload_1
20: isub
21: ireturn
\end{lstlisting}
\end{lrbox}

\newsavebox{\optimized}
\begin{lrbox}{\optimized}
\begin{lstlisting}[showlines=true]
0: sipush	3630
21: ireturn














\end{lstlisting}
\end{lrbox}

\begin{figure}[h!]
	\centering
	\subfloat[Unoptimized]{\usebox{\unoptimized}}
	\hspace{0.5cm}\vline\hspace{0.5cm}
	\subfloat[Optimized]{\usebox{\optimized}}
	\caption{Unoptimized and optimized Java bytecode}
\end{figure}